% Created 2017-10-15 dim. 17:11
\documentclass[12pt,a4paper,twoside]{article}
 \usepackage[T1]{fontenc}
\usepackage[utf8]{inputenc}
\usepackage{amsmath,amssymb,gensymb,mathtools}
\usepackage{xspace}
\usepackage[paper=a4paper,textwidth=6.6in,top=38mm,bottom=38mm]{geometry} % ,left=20mm,right=20mm, ,headsep=0pt
%\usepackage{a4wide}
\usepackage{palatino}
\usepackage{verbments}
\plset{numbers=left,numbersep=5pt,texcl=true,style=tango,bgcolor=Moccasin,fontsize=\small}
\usepackage{xcolor}
\usepackage{todonotes}
\usepackage{color}
\usepackage[normalem]{ulem}
\usepackage{fancyhdr}
\AtBeginDocument{
\definecolor{pdfurlcolor}{rgb}{0,0,0.6}
\definecolor{pdfcitecolor}{rgb}{0,0.6,0}
\definecolor{pdflinkcolor}{rgb}{0.6,0,0}
\definecolor{light}{gray}{.85}
\definecolor{vlight}{gray}{.95}
\selectlanguage{francais}
}
\usepackage{url} \urlstyle{sf}
\usepackage[francais, english]{babel}
\selectlanguage{francais}
\let\oldmaketitle=\maketitle
\def\maketitle{}
\usepackage[colorlinks=true,citecolor=pdfcitecolor,urlcolor=pdfurlcolor,linkcolor=pdflinkcolor,pdfborder={0 0 0}]{hyperref}
\pagestyle{fancy}\fancyhead{}\fancyfoot{}
\usepackage{subfigure}
%\usepackage[nomarkers,figuresonly]{endfloat}\renewcommand{\efloatseparator}{\mbox{}}



\usepackage{minted}
\setcounter{secnumdepth}{4}
\author{Clément Legrand}
\date{2017}
\title{Rapport}
\hypersetup{
 pdfauthor={Clément Legrand},
 pdftitle={Rapport},
 pdfkeywords={},
 pdfsubject={},
 pdfcreator={Emacs 24.5.1 (Org mode 8.3.4)}, 
 pdflang={English}}
\begin{document}

\maketitle
\renewcommand{\v}[1]{\ensuremath{\overrightarrow{#1}}\xspace}
\let\oldcite=\cite
\def\cite#1{~\oldcite{#1}\xspace}
\let\oldref=\ref
\def\ref#1{~\oldref{#1}\xspace}
\let\oldeqref=\eqref
\def\eqref#1{~\oldeqref{#1}\xspace}
\let\leq=\leqslant
\let\geq=\geqslant
\let\le=\leqslant
\let\ge=\geqslant
\def\R{\ensuremath{\mathbb{R}}\xspace}
\pagestyle{empty} 
\pagenumbering{gobble}
\let\maketitle=\oldmaketitle

\pagestyle{fancy}
\fancyhead{}
\fancyfoot{}
\rhead[\sffamily\itshape \MakeUppercase{Rapport}]{\thepage}
\lhead[\thepage]{\sffamily\itshape \leftmark}
\pagenumbering{roman}
\pagenumbering{arabic}

\begin{center}
\bgroup\bf \LARGE Rapport\egroup\medskip

\large Clément Legrand \smallskip

\normalsize L3 Info ENS, 2017/2018
\end{center}

\section{Introduction}
\label{sec:orgheadline1}
Seven colors est un jeu d'ordinateur inventé par Dmitry Pashkov,
dévellopé en 1991. Le principe est simple: deux joueurs essayent de
conquérir la plus grande surface d'un terrain de jeu consistué de
cases juxtaposées, de sept couleurs différentes. Pour cela, chaque
joueur chosit à son tour une couleur, et conquiert toute case de
cette couleur adjacente à la zone qu'il posséde.

\section{Réponses aux questions}
\label{sec:orgheadline3}
\subsection{Question 1}
\label{sec:orgheadline2}
Il convient tout d'abord de créer un type \texttt{boardd} afin
d'encapsuler un peu le code et de ne pas modifier le pateau de jeu
n'importe comment. Ce type contient un tableau de char de taille
\(\texttt{BOARDSIZE} ^{2}\), ainsi que deux entiers:
\texttt{numcellsup} et \texttt{numcellsdown} indiquant le nombre de
cases possédées par le joueur \^{} et le joueur v respectivement.

\section{Conclusion}
\label{sec:orgheadline4}
\label{sec:conclusion}
blabla
\clearpage\appendix

\section{Annexe}
\label{sec:orgheadline6}
\subsection{Bibliographie}
\label{sec:orgheadline5}
Ne sont cités ici que les ouvrages et sites dont je me suis le plus
servi.

\def\section*#1{}
\bibliographystyle{abbrv-fr}
\nocite{*}
\bibliography{biblio.bib}
\clearpage
\end{document}