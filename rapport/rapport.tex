% Created 2017-10-15 dim. 21:41
\documentclass[12pt,a4paper,twoside]{article}
 \usepackage[T1]{fontenc}
\usepackage[utf8]{inputenc}
\usepackage{amsmath,amssymb,gensymb,mathtools}
\usepackage{xspace}
\usepackage{listings}%

\usepackage[paper=a4paper,textwidth=6.6in,top=38mm,bottom=38mm]{geometry} % ,left=20mm,right=20mm, ,headsep=0pt
%\usepackage{a4wide}
\usepackage{palatino}
\usepackage{verbments}
\plset{numbers=left,numbersep=5pt,texcl=true,style=tango,bgcolor=Moccasin,fontsize=\small}
\usepackage{xcolor}
\usepackage{todonotes}
\usepackage{color}
\usepackage[normalem]{ulem}
\usepackage{fancyhdr}
\AtBeginDocument{
\definecolor{pdfurlcolor}{rgb}{0,0,0.6}
\definecolor{pdfcitecolor}{rgb}{0,0.6,0}
\definecolor{pdflinkcolor}{rgb}{0.6,0,0}
\definecolor{light}{gray}{.85}
\definecolor{vlight}{gray}{.95}
\selectlanguage{francais}
}
\usepackage{url} \urlstyle{sf}
\usepackage[francais, english]{babel}
\selectlanguage{francais}
\let\oldmaketitle=\maketitle
\def\maketitle{}
\usepackage[colorlinks=true,citecolor=pdfcitecolor,urlcolor=pdfurlcolor,linkcolor=pdflinkcolor,pdfborder={0 0 0}]{hyperref}
\pagestyle{fancy}\fancyhead{}\fancyfoot{}
\usepackage{subfigure}
%\usepackage[nomarkers,figuresonly]{endfloat}\renewcommand{\efloatseparator}{\mbox{}}



\usepackage{minted}
\setcounter{secnumdepth}{4}
\author{Marco Freire et Clément Legrand}
\date{2017}
\title{Rapport du projet 7 colors}
\hypersetup{
 pdfauthor={Marco Freire et Clément Legrand},
 pdftitle={Rapport du projet 7 colors},
 pdfkeywords={},
 pdfsubject={},
 pdfcreator={Emacs 24.5.1 (Org mode 8.3.4)}, 
 pdflang={English}}
\begin{document}

\maketitle
\renewcommand{\v}[1]{\ensuremath{\overrightarrow{#1}}\xspace}
\let\oldcite=\cite
\def\cite#1{~\oldcite{#1}\xspace}
\let\oldref=\ref
\def\ref#1{~\oldref{#1}\xspace}
\let\oldeqref=\eqref
\def\eqref#1{~\oldeqref{#1}\xspace}
\let\leq=\leqslant
\let\geq=\geqslant
\let\le=\leqslant
\let\ge=\geqslant
\def\R{\ensuremath{\mathbb{R}}\xspace}
\pagestyle{empty} 
\pagenumbering{gobble}
\let\maketitle=\oldmaketitle

\pagestyle{fancy}
\fancyhead{}
\fancyfoot{}
\rhead[\sffamily\itshape \MakeUppercase{Rapport}]{\thepage}
\lhead[\thepage]{\sffamily\itshape \leftmark}
\pagenumbering{roman}
\pagenumbering{arabic}

\begin{center}
\bgroup\bf \LARGE Rapport du projet 7 colors\egroup\medskip

\large Marco Freire et Clément Legrand \smallskip

\normalsize L3 Info ENS, 2017/2018
\end{center}

\section{Introduction}
\label{sec:orgheadline1}
Seven colors est un jeu d'ordinateur inventé par Dmitry Pashkov,
développé en 1991. Le principe est simple: deux joueurs essayent de
conquérir la plus grande surface d'un terrain de jeu constitué de
cases juxtaposées, de sept couleurs différentes. Pour cela, chaque
joueur choisit à son tour une couleur, et conquiert toute case de
cette couleur adjacente à la zone qu'il possède.  Nous nous sommes
penchés sur l'implémentation de ce jeu, de ses mécanismes et de
différents joueurs artificiels. Ce projet amène à comparer ces
différents joueurs entre eux afin de rechercher une stratégie aussi
bonne que possible. De plus, de part la taille du programme, il
devient réellement nécessaire de modulariser le code et de le
découper en plusieurs fichiers.
\section{Réponses aux questions}
\label{sec:orgheadline25}
\subsection{Voir le monde en sept couleurs}
\label{sec:orgheadline9}
\subsubsection*{Implémentation du type \texttt{board\_d}}
Il convient tout d'abord de créer un type \texttt{board\_d} afin
d'encapsuler un peu le code et de ne pas modifier le plateau de jeu
n'importe comment. Ce type contient un tableau de caractères de taille
\(\texttt{BOARD\_SIZE} ^{2}\), ainsi que deux entiers:
\texttt{num\_cell\_sup} et \texttt{num\_cells\_down} indiquant le nombre de
cases possédées par le joueur \texttt{\^} et le joueur \texttt{v} respectivement.
Afin de maintenir cette structure, nous avons implémenté les
méthodes \texttt{get\_cell}, \texttt{get\_num\_cells\_up},
\texttt{get\_num\_cells\_down}, \texttt{set\_cell}, ainsi que
\texttt{board\_create} et \texttt{board\_free}, allouant un tableau de
la bonne taille, et le libérant respectivement. 
\subsubsection*{Question 1}
\label{sec:orgheadline2}
La fonction \texttt{rand\_board} prend en argument un pointeur vers un
tableau, et initialise en place chacune des cases de celui-ci
aléatoirement.  Nous avons pour cela recours à la fonction
\texttt{rand}. Celle-ci renvoie un \texttt{int} tiré uniformement (et
donc compris entre \(0\)
et \(RAND\_MAX = 32767\)),
que nous ramenons entre 0 et 6 avec une division euclidienne. On
notera que 32768 n'est pas divsible par 7, ainsi la lettre A est très
légerement plus probable que les autres, mais ceci est négligeable
(aux vues des dimensions du tableaux notamment).
\subsubsection*{Question 2}
\label{sec:orgheadline5}
La fonction \texttt{update\_board} prend en argument un pointeur vers
le tableau de jeu, ainsi que le joueur courant et la couleur jouée.
la fonction \texttt{print\_board} affiche les valeurs des cases du
tableau du pointeur qui lui est donné en argument. Elle permet de
s'assurer sur quelques exemples que le tableau évolue bien de la
manière souhaitée. 
\paragraph*{Complexité dans le pire cas :}
\label{sec:orgheadline3}
Dans le pire des cas, à chaque itération, l'algorithme met à jour une
et une seule case. Posons \(n = \texttt{BOARD\_SIZE}\). Étant donné qu'il y a \(n^{2}\)
cases dans le tableau, au plus \(n^{2}\) passages sont effectués. Lors d'un
passage, \(n^{2}\) cases sont examinées, chacune d'elles en un temps
constant. La complexité dans le pire cas de cet algorithme est donc un
\(O(n^4)\).
Cette borne est réellement atteinte: par exemple si une couleur forme
une spirale rectangulaire jusqu'au centre. La complexité est donc un
\(\Theta(n^{4})\). 
\paragraph*{Complexité amortie :}
\label{sec:orgheadline4}
Le cas énoncé précédemment est assez pathologique et ne peut se
produire trop souvent. L'analyse de la complexité amortie serait donc
pertinente, bien qu'un peu délicate.
\subsubsection*{Question 3 (bonus)}
\label{sec:orgheadline8}
\paragraph*{Principe de l'algorithme implémenté :}
\label{sec:orgheadline6}
Plutôt que d'effectuer des parcours entiers du tableau en vérifiant si
les cases ont besoin d'être mises à jour, il est préférable de se
représenter la zone conquise par un joueur comme un graphe (non
orienté), dont les sommets sont les toutes les cases du tableau et les
arêtes lient chacune des cases conquises par le joueur aux quatre
cases adjacentes sur le plateau. Il suffit alors d'effectuer un
parcours de ce graphe en le mettant a jour progressivement. 

Nous implémentons donc deux fonctions, la première
\texttt{update\_board\_optimized} crée un second tableau
\texttt{board\_visited}, de mêmes dimensions, dont les cases indiquent
si le sommet \((i,j)\)
a été visité ou pas.  Nous parcourons ensuite récursivement le graphe
en commençant par la case du coin (qui appartient nécessairement au
joueur), à l'aide de la fonction \texttt{update\_board\_rec}.  À
chaque sommet visité, nous le marquons comme tel dans
\texttt{board\_visited}, et s'il est de la bonne couleur ou appartient
au joueur courant, nous marquons la case comme appartenant au joueur
dans \texttt{board} et explorons ses voisines non visitées.

Les fonctions \texttt{update\_board\_rec} et
\texttt{update\_board\_optimized} opèrent toutes deux en place.
\paragraph*{Complexité dans le pire des cas :}
\label{sec:orgheadline7}
Chaque case du tableau est visité un nombre fini borné de fois (au
plus quatre dans notre implémentation), donc l'algorithme est un
\(O(n^{2})\). 
Là encore, cette borne peut être atteinte si tout le tableau
est de la même couleur. La complexité est donc un \(\Theta(n^{2})\).
\subsection{À la conquête du monde}
\label{sec:orgheadline12}
\subsubsection*{Question 4}
\label{sec:orgheadline10}
Les fonctions nécessaires pour faire jouer un joueur contre un autre sont
\texttt{player\_input} qui permet de récupérer la couleur jouée par le
joueur courant, et \texttt{change\_player}. La fonction
\texttt{game\_loop} effectue alors l'essentiel du travail en faisant
appel aux bonnes fonctions. Pour jouer à deux, il suffit de remplacer

\begin{lstlisting}
case 'v':
    color_input = IA_random_wise(board, *player);
    *is_IA_turn = true;
    break;
\end{lstlisting}
par 
\begin{lstlisting}
case 'v':
    color_input = player_input();
    *is_IA_turn = true;
    break;
\end{lstlisting}


L'implémentation contraint les joueurs à s'affronter sur la même
machine.

\subsubsection*{Question 5}
\label{sec:orgheadline11}
La partie peut s'arrêter si l'un des joueurs au moins détient plus de
la moitié du plateau de jeu. Nous avons donc recours aux attributs
\texttt{num\_cells\_up} et \texttt{num\_cells\_down} du type \texttt{board\_d},
afin de savoir si ceux ci excèdent la moitié du nombre total de cases.
La fonction \texttt{is\_end} vérifie cette condition, et la fonction
\texttt{print\_occupation} affiche le pourcentage du plateau occupé
par chaque joueur.

\subsection{La stratégie de l'aléa}
\label{sec:orgheadline15}
\subsubsection*{Question 6}
\label{sec:orgheadline13}
La fonction \texttt{IA\_random} ne prend aucun argument et renvoie une
couleur tirée au hasard.
\subsubsection*{Question 7}
\label{sec:orgheadline14}
La fonction \texttt{IA\_random\_wise} fait de même en s'assurant que
que la couleur jouée permet de gagner du terrain. On détermine pour
cela quelles sont les couleurs admissibles, et nous tirons ensuite
une couleur au hasard, jusqu'à que celle-ci soit admissible.
\subsection{La loi du plus fort}
\label{sec:orgheadline19}
\subsubsection*{Question 8}
\label{sec:orgheadline16}
La fonction \texttt{IA\_greedy} prend en argument un pointeur vers
le plateau de jeu, ainsi que le joueur courant et renvoie la couleur
choisie par l'algorithme glouton.  Nous avons pour cela repris et
légèrement modifié la fonction \texttt{update\_board\_optimized}
impémentée pour la question 3: nous comptons désormais le nombre de
cases modifiables en jouant une couleur donnée au lieu de les
modifier. Nous effectuons ensuite le choix maximisant ce nombre de
cases.
\subsubsection*{Question 9}
\label{sec:orgheadline17}
Pour que le combat soit "équitable", le mieux est de faire s'affronter
les deux joueurs artificiels sur un plateau donné, puis d'effectuer
l'affrontement à nouveau, mais en échangeant les positions de départ.
\subsubsection*{Question 10}
\label{sec:orgheadline18}
Sur les cent essais effectués (sur des grilles carées de coté 30, et
en échangeant à chaque fois les rôles des joueurs), le
joueur artificiel glouton à systématiquement gagné. 
Il pourrait être intéressant de déterminer expérimentalement
l'influence de la taille de la grille.
\subsection{Les nombreuses huitièmes merveilles du monde (bonus)}
\label{sec:orgheadline22}
\subsubsection*{Question 11}
\label{sec:orgheadline20}
Il apparaît facilement qu'en effectuant un parcours entier du plateau
et en comptant les cases ayant parmi leurs voisines, une appartenant au
joueur courant, il est possible d'obtenir le périmètre de la zone
détenue par le joueur en \(\Theta(n^{2})\). Il suffit dès lors, pour chaque
couleur possible, de créer une copie du plateau, d'effectuer sur celle-
ci une mise à jour en jouant ladite couleur, et de compter le
périmètre correspondant.
La complexité de cet algorithme est donc un \(\Theta(n^{2})\).
La fonction modélisant l'algorithme hégémonique est

\texttt{IA\_greedy\_border}.
\subsubsection*{Question 12}
\label{sec:orgheadline21}
La fonction modélisant le glouton prévoyant est
\texttt{IA\_foresighted\_greedy}. 
Une des manières d'implémenter le glouton prévoyant est d'effectuer
pour chacune des couleurs une copie du plateau de jeu, puis de
modifier celui-ci en jouant cette couleur, en comptant le nombre de
cases ainsi gagnées. Puis d'utiliser la fonction de la question 8 pour
chacune des couleurs afin d'obtenir le nombre maximal de cases gagnées
en ayant fixé le premier coup. On choisit ensuite le premier coup afin
de maximiser ce nombre.  Cet algorithme effectue 7 copies du plateau
en \(\Theta(n^{2})\), fait sept appels à \texttt{update\_board\_optimized} en
\(\Theta(n^{2})\) et fait appel 49 fois à la fonction
\texttt{calc\_new\_cells\_optimized} qui s'exécute en \(\Theta(n^{2})\). Il s'exécute
donc en \(\Theta(n^{2})\).
Si on veut explorer \(m\) coups consécutifs, cette complexité passe à un
\(\Theta(\exp(m)*n^{2})\), car il faut alors effectuer 7 fois plus de copies du
tableau pour chaque coup additionnel (sauf le dernier ou il est possible
d'utiliser \texttt{calc\_new\_cells\_optimized}).

\subsection{Le pire du monde merveilleux des sept couleurs (bonus)}
\label{sec:orgheadline24}
\subsubsection*{Question 14}
\label{sec:orgheadline23}
Cet algorithme n'a pas été implémenté. 

En revanche, on remarque que l'évolution du plateau de jeu est
radicalement différente entre le joueur artificiel hégémonique et le
glouton. De plus, l'efficacité de cette évolution varie grandement en
fonction de l'avancée du jeu.  En effet, l'algorithme du glouton
hégémonique va avoir tendance à étendre la zone maîtrisée par le
joueur en se ramifiant énormément, quitte à laisser des ''trous'' dans
la zone conquise. Par ailleurs, cet algorithme semble extrêmement
efficace en début de partie, au contraire, la fin de partie est très
lente : si la zone maitrisée englobe totalement une zone non conquise
par exemple, alors conquérir les cases au bord de celle-ci réduit le
périmètre, cette conquête va donc avoir tendance à être ralentie.  À
l'inverse la zone maitrisée par l'algorithme glouton semble être plus
compacte et évoluer un peu plus lentement.

De ces observations, il paraît judicieux d'utiliser l'algorithme du
glouton hégémonique jusqu'à un certain point, avant de changer de
stratégie, et d'utiliser l'algorithme glouton classique ou prévoyant.

Reste à déterminer le moment judicieux pour effectuer la
transition. Cela peut se déterminer expérimentalement en traçant la
vitesse d'expansion de la zone maîtrisée par l'algorithme hégémonique
en fonction du temps. Ou bien en traçant celle ci en fonction du
pourcentage d'occupation du terrain. Ou encore en choisissant de
déterminer le tour de transition à l'aide d'une loi de Poisson par
exemple, dont le paramètre \(\lambda\) optimal, serait recherché
expérimentalement.
 
\section{Synthèse}
\label{sec:orgheadline26}
\label{sec:conclusion}
Plusieurs joueurs artificiels ont été implémentés: 
\begin{itemize}
\item l'algorithme glouton classique consistant à maximiser le nombre
  de cases gagnées à chaque tour,
\item l'algorihtme glouton prévoyant, faisant le choix permettant de
  maximiser le nombre de cases gagnées au terme de deux tours,
\item l'algorithme glouton hégémonique, maximisant le périmètre de la
  zone maîtrisée.
\end{itemize}


\clearpage\appendix

\section{Annexe}
\label{sec:orgheadline28}
\subsection{Bibliographie}
\label{sec:orgheadline27}
Ne sont cités ici que les ouvrages et sites dont nous nous sommes le plus
servi.

\def\section*#1{}
\bibliographystyle{abbrv-fr}
\nocite{*}
\bibliography{biblio.bib}
\clearpage
\end{document}
